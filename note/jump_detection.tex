\documentclass[11pt,a4paper]{article}

\usepackage{style}

\begin{document}
\title{Jump Detection}
%\author{Hyungjin Kim
%\\{\it\small Deutsches Elektronen-Synchrotron DESY, Notkestr. 85, 22607 Hamburg, Germany}}
\maketitle

Here we consider time series and ways to detect discontinuities or sudden jumps in such time series. We begin by reviewing a wavelet decomposition of time series data. For details, see Ref.~\cite{1992tlw..conf.....D}.

\section{Wavelet}
\subsection{Continuous Wavelet}
What is a wavelet? A function $\psi(x)$ is called a wavelet if it satisfies
\begin{align}
&
\int_{-\infty}^{\infty} dx \, \psi(x)  = 0
\\
&
\int_{-\infty}^{\infty} \frac{d\omega}{2\pi} \, \frac{|\psi(\omega)|^2}{\omega} = C_\psi < 0
\end{align}
Here $C_\psi$ is called admissible constant. A wavelet family can be generated from the above mother wavelet by translation and dilation
\begin{align}
\psi_{a,b}(x) 
= \frac{1}{\sqrt{a}} \psi \left( \frac{x - b}{a} \right)
\end{align}
The parameter $a$ controls the dilation, while $b$ controls the shift of the wavelet. 

The {\it continuous wavelet transformation} (CWT) of a function $f(x)$ is defined by
\begin{align}
W f(a,b) = \int_{-\infty}^\infty dx \,  f(x) \psi^*_{a,b} (x)
\equiv \langle f, \psi_{a,b}  \rangle
\end{align}
The angle bracket denotes $L^2$-inner product. The inverse transformation is defined as
\begin{align}
f(x) 
= 
\frac{1}{C_\psi} 
\int_{-\infty}^\infty \frac{da}{a^2}
\int_{-\infty}^{\infty} db \, 
Wf(a,b) \, \psi_{a,b}(x).
\end{align}
This continuous wavelet decomposition can be used to detect jump. Consider a Heaviside step function $H(x)$. The continuous wavelet transformation of $H(x)$ is
\begin{align}
W f (a,b) 
&= \int_{-\infty}^{\infty} \, dx \, H(x) \psi^*_{a,b}(x)
= \int_{0}^{\infty} \, \frac{dx}{\sqrt{a}} \psi^*\Big( \frac{x-b}{a} \Big).
\end{align}
From the above expression, we see that $Wf(a,0) / \sqrt{a}$ takes a constant value regardless of the value $a$. This indicates a discontinuity at $b=0$. In general, if $Wf(a,b) / \sqrt{a}$ takes a constant value for all $a$ at a particular $b$, this signals a discontinuity of the original function $f(x)$ at $x=b$. 





%\subsection{Examples of Wavelet}
%\subsubsection{Haar}

\subsection{Discrete Wavelet}
In the discrete case, a family of wavelets is constructed from the mother wavelet via
\begin{align}
\psi_{j,k}(x) = 2^{-j / 2} \psi( 2^{-j} x - k)
\end{align}
for $j, k \in {\mathbb Z}$. The discrete wavelet transformation is defined as
\begin{align}
W f_{j,k} = \langle f,\, \psi_{j,k}  \rangle
= \int_{-\infty}^{\infty} dx \, \psi^*_{j,k}(x) f(x) 
\end{align}
The inverse transformation is 
\begin{align}
f(x) = \sum_{j,k} \psi_{j,k}(x) Wf_{j,k}
\end{align}



\appendix
\section{}

The jump time series is defined as
\begin{align}
x_t
= 
\frac{r_t}{f_t \sigma_t}
\end{align}
where $r_t = \ln( p_t / p_{t-1})$ is 1-minute return, $f_t$ is an estimator of intraday periodicity, and $\sigma_t$ is an estimator for the local volatility. The $\sigma_t$ is defined as
\begin{align}
\sigma_t^2
= \frac{\pi}{2 K}
\sum_{i=1}^{390} 
|r_{t-i}| |r_{t-i+1}|
\end{align}



\bibliographystyle{utphys}
\bibliography{ref}
\end{document}


