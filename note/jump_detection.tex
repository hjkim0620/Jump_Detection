\documentclass[11pt,a4paper]{article}

\usepackage{style}

\begin{document}
\title{Jump Detection}
%\author{Hyungjin Kim
%\\{\it\small Deutsches Elektronen-Synchrotron DESY, Notkestr. 85, 22607 Hamburg, Germany}}
\maketitle

The jump time series is defined as
\begin{align}
x_t
= 
\frac{r_t}{f_t \sigma_t}
\end{align}
where $r_t = \ln( p_t / p_{t-1})$ is 1-minute return, $f_t$ is an estimator of intraday periodicity, and $\sigma_t$ is an estimator for the local volatility. The $\sigma_t$ is defined as
\begin{align}
\sigma_t^2
= \frac{\pi}{2 K}
\sum_{i=1}^{390} 
|r_{t-i}| |r_{t-i+1}|
\end{align}


\section{Wavelet}
\subsection{Continuous Wavelet}
What is a wavelet? A function $\psi(x)$ is called a wavelet if it satisfies
\begin{align}
&
\int_{-\infty}^{\infty} dx \, \psi(x)  = 0
\\
&
\int_{-\infty}^{\infty} d\omega \, \frac{|\psi(\omega)|^2}{\omega} = C_\psi < 0
\end{align}
Here $C_\psi$ is called admissible constant. A wavelet family can be generated from the above mother wavelet by translation and dilation
\begin{align}
\psi_{a,b}(x) 
= \frac{1}{\sqrt{a}} \psi \left( \frac{x - b}{a} \right)
\end{align}
The parameter $a$ controls the dilation, while $b$ controls the shift of the wavelet. 

The {\it continuous wavelet transformation} of a function $f(x)$ is defined by
\begin{align}
W f(a,b) = \int_{-\infty}^\infty dx \, f(x) \psi_{a,b} (x)
\end{align}
while the inverse transformation is defined as

%\subsection{Examples of Wavelet}
%\subsubsection{Haar}

\subsection{Discrete Wavelet}
In the discrete case, a family of wavelets is constructed from the mother wavelet via
\begin{align}
\psi_{j,k}(x) = 2^{-j / 2} \psi( 2^{-j} x - k)
\end{align}
for $j, k \in {\mathbb Z}$. 



\bibliographystyle{utphys}
\bibliography{ref}
\end{document}


